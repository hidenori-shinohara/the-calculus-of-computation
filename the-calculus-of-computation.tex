\documentclass[12pt, psamsfonts]{amsart}

%-------Packages---------
\usepackage{amssymb,amsfonts}
\usepackage{xcolor,colortbl}
\usepackage{semantic}
\usepackage{fullpage}
\usepackage{units}
\usepackage{tikz-cd}
\usepackage{todonotes}
\usepackage{physics}
\usepackage[all,arc]{xy}
\usepackage{enumerate}
\usepackage{enumitem}
\usepackage{mathrsfs}
\usepackage{theoremref}
\usepackage{graphicx}
\usepackage[bookmarks]{hyperref}

\usepackage{amsthm}
\makeatletter
\def\th@plain{%
  \thm@notefont{}% same as heading font
  \itshape % body font
}
\def\th@definition{%
  \thm@notefont{}% same as heading font
  \normalfont % body font
}
\makeatother

%--------Theorem Environments--------
%theoremstyle{plain} --- default
\newtheorem{thm}{Theorem}[subsection]
\newtheorem{cor}[thm]{Corollary}
\newtheorem{prop}[thm]{Proposition}
\newtheorem{lem}[thm]{Lemma}
\newtheorem{conj}[thm]{Conjecture}
\newtheorem{quest}[thm]{Question}

\theoremstyle{definition}
\newtheorem{defn}[thm]{Definition}
\newtheorem{defns}[thm]{Definitions}
\newtheorem{con}[thm]{Construction}
\newtheorem{exmp}[thm]{Example}
\newtheorem{exmps}[thm]{Examples}
\newtheorem{notn}[thm]{Notation}
\newtheorem{notns}[thm]{Notations}
\newtheorem{addm}[thm]{Addendum}
\newtheorem*{exer}{Exercise}

\theoremstyle{remark}
\newtheorem{rem}[thm]{Remark}
\newtheorem{rems}[thm]{Remarks}
\newtheorem{warn}[thm]{Warning}
\newtheorem{sch}[thm]{Scholium}

\DeclareMathOperator{\weight}{weight}
\DeclareMathOperator{\neighbors}{neighbors}
\DeclareMathOperator{\priority}{priority}
\newcommand{\textoverline}[1]{$\overline{\mbox{#1}}$}


\makeatletter
\makeatother
\numberwithin{equation}{subsection}

\bibliographystyle{plain}

\setcounter{tocdepth}{3}
\makeatletter
\def\l@subsection{\@tocline{2}{0pt}{2.5pc}{5pc}{}}
\makeatother

\begin{document}

\title{The Calculus of Computation}
\author{Hidenori Shinohara}

\maketitle

\section{Chapter 1}

\begin{exer}[1.1]
    $ $
    \begin{enumerate}[label=(\alph*)]
        \item
            Assume that there is a falsifying interpretation $I$.
            \begin{enumerate}[label=\arabic*.]
                \item % 1
                    $I \models P \land Q \rightarrow P \rightarrow Q$ (assumption)
                \item % 2
                    $I \models P \land Q$ (by 1 and semantics of $\rightarrow$)
                \item % 3
                    $I \not\models P \rightarrow Q$ (by 1 and semantics of $\rightarrow$)
                \item % 4
                    $I \models Q$ (by 2 and semantics of $\land$)
                \item % 5
                    $I \not\models Q$ (by 3 and semantics of $\rightarrow$)
                \item % 6
                    $I \models \bot$ (4 and 5 are contradictory)
            \end{enumerate}
            There is only one branch and it is closed.
            Thus $F$ is valid.
    \end{enumerate}
\end{exer}

\end{document}
