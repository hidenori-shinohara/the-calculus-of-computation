\begin{exer}[1.1]
    $ $
    \begin{enumerate}[label=(\alph*)]
        \item
            Assume that there is a falsifying interpretation $I$.
            \begin{enumerate}[label=\arabic*.]
                \item % 1
                    $I \not\models P \land Q \rightarrow P \rightarrow Q$ (assumption)
                \item % 2
                    $I \models P \land Q$ (by 1 and semantics of $\rightarrow$)
                \item % 3
                    $I \not\models P \rightarrow Q$ (by 1 and semantics of $\rightarrow$)
                \item % 4
                    $I \models Q$ (by 2 and semantics of $\land$)
                \item % 5
                    $I \not\models Q$ (by 3 and semantics of $\rightarrow$)
                \item % 6
                    $I \models \bot$ (4 and 5 are contradictory)
            \end{enumerate}
            There is only one branch and it is closed.
            P.13 states that it is a proof of the validity of $F$ if every branch is closed.
            Thus $F$ is valid.
        \item
            By the relative precedence of the logical connectives mentioned on P.5, the given formula is the same as
            \begin{align*}
                F: (P \rightarrow Q) \lor (P \land \neg Q).
            \end{align*}
            Suppose that there is a falsifying interpretation $I$.
            We will use the semantics mentioned on P.10 and P.11.
            \begin{enumerate}[label=\arabic*.]
                \item % 1
                    $I \not\models (P \rightarrow Q) \lor (P \land \neg Q)$ (assumption)
                \item % 2
                    $I \not\models P \rightarrow Q$ (by 1 and semantics of $\lor$)
                \item % 3
                    $I \not\models P \land \neg Q$ (by 1 and semantics of $\lor$)
                \item % 4
                    $I \models P$ (by 2 and semantics of $\rightarrow$)
                \item % 5
                    $I \not\models Q$ (by 2 and semantics of $\rightarrow$)
            \end{enumerate}
            By 3 and the semantics of $\land$, we obtain two branches.

            The first branch is:
            \begin{enumerate}[label=\arabic*a.]
                \item % 1a
                    $I \not\models P$ (by 3 and semantics of $\land$)
                \item % 2a
                    $I \models \bot$ (by 4 and 1a)
            \end{enumerate}

            The second branch is:
            \begin{enumerate}[label=\arabic*b.]
                \item % 1b
                    $I \not\models \neg Q$ (by 3 and semantics of $\land$)
                \item % 2b
                    $I \models Q$ (by 1b and $\neg$)
                \item % 3b
                    $I \models \bot$ (by 5 and 2b)
            \end{enumerate}

            Every branch is closed, so this is a proof of validity of $F$.
        \item
            We claim that the interpretation $I: \{ P \mapsto \textsf{true}, Q \mapsto \textsf{false}, R \mapsto \textsf{false} \}$ is a falsifying interpretation.
            We will use the inductive definition on P.7 and P.8.
            \begin{align*}
                &I \not\models (P \rightarrow Q \rightarrow R) \rightarrow P \rightarrow R \\
                    &\text{iff}\quad I \models P \rightarrow Q \rightarrow R \text{ and } I \not\models P \rightarrow R \\
                    &\text{iff}\quad I \models P \rightarrow Q \rightarrow R \text{ and } I \models P \text{ and } I \not\models R \\
                    &\text{iff}\quad I \models P \rightarrow Q \rightarrow R \text{ and } I[P] = \textsf{true} \text{ and } I[R] = \textsf{false} \\
                    &\text{iff}\quad \text{if } I \models P \text{, then } I \models Q \rightarrow R \\
                    &\text{iff}\quad \text{if } I[P] = \textsf{true} \text{, then } I \models Q \rightarrow R \\
                    &\text{iff}\quad I \models Q \rightarrow R \\
                    &\text{iff}\quad \text{if } I \models Q \text{, then } I \models R \\
                    &\text{iff}\quad \text{if } I[Q] = \textsf{true} \text{, then } I \models R \\
                    &\text{iff}\quad I \models R \\
                    &\text{iff}\quad I[R] = \textsf{true}.
            \end{align*}
            Therefore, $I$ is indeed a falsifying interpretation of the given formula.
        \item
            We claim that the interpretation $I: \{ P \mapsto \textsf{true}, Q \mapsto \textsf{true}, R \mapsto \textsf{false} \}$ is a falsifying interpretation.
            We will use the inductive definition on P.7 and P.8.
            \begin{align*}
                &I \not\models (P \rightarrow Q \lor R) \rightarrow P \rightarrow R \\
                    &\text{iff}\quad I \models P \rightarrow Q \lor R \text{ and } I \not\models P \rightarrow R \\
                    &\text{iff}\quad I \models P \rightarrow Q \lor R \text{ and } I \models P \text{ and } I \not\models R \\
                    &\text{iff}\quad I \models P \rightarrow Q \lor R \\
                    &\text{iff}\quad \text{ if } I \models P \text{ then } I \models Q \lor R \\
                    &\text{iff}\quad \text{ if } I \models P \text{ then } (I \models Q \text{ or } I \models R)
            \end{align*}
            where the last statement is true since $I \models Q$.
            Therefore, $I$ is indeed a falsifying interpretation of the given formula.
        \item
            Assume that there is a falsifying interpretation $I$.
            \begin{enumerate}[label=\arabic*.]
                \item % 1
                    $I \not\models \neg(P \land Q) \rightarrow R \rightarrow \neg R \rightarrow Q$.
                \item % 2
                    $I \models P \land Q$.
                \item % 3
                    $I \models R \rightarrow \neg R \rightarrow Q$.
                \item % 4
                    $I \models R$.
                \item % 5
                    $I \not\models \neg R \rightarrow Q$.
                \item % 6
                    $I \models R$.
            \end{enumerate}
            4 and 6 lead to a contradiction.
        \item
            By the relative precedence of the logical connectives on P.5, we have $(P \land Q) \lor \neg P \lor (\neg Q \rightarrow \neg P)$.
            We claim that $I: \{ P \mapsto \textsf{true}, Q \mapsto \textsf{false} \}$ is a falsifying interpretation.
            We will use the inductive definition on P.7 and P.8.
            \begin{align*}
                &I \not\models (P \land Q) \lor \neg P \lor (\neg Q \rightarrow \neg P) \\
                    &\text{iff}\quad I \not\models P \land Q \text{ and } I \not\models \neg P \text{ and } I \not\models \neg Q \rightarrow \neg P \\
                    &\text{iff}\quad I \not\models P \land Q \text{ and } I \models P \text{ and } I \models \neg Q \text{ and } I \not\models\neg P \\
                    &\text{iff}\quad I \not\models P \land Q
            \end{align*}
            We know that $I \not\models P \land Q$ by $I \not\models Q$ and the semantics of $\land$.
            (See Example 1.4)
            Therefore, $I$ is indeed a falsifying interpretation, and thus the given formula is invalid.
    \end{enumerate}
\end{exer}

\begin{exer}[1.2]
    $ $
    \begin{enumerate}[label=(\alph*)]
        \item
            To prove that $\top \Leftrightarrow \neg \bot$, we prove that $\top \leftrightarrow \neg \bot$ is valid.
            Assume that there is a falsifying interpretation $I$ such that $I \not\models \top \leftrightarrow \neg \bot$.
            We apply the semantics of $\leftrightarrow$.

            The first branch is:
            \begin{enumerate}[label=\arabic*a.]
                \item % 1a
                    $I \models \top \land \neg(\neg \bot)$ 
                \item % 2a
                    $I \models \neg(\neg \bot)$ (by 1a and semantics of $\land$)
                \item % 3a
                    $I \not\models \neg \bot$ (by 2a and semantics of $\neg$)
                \item % 4a
                    $I \models \bot$ (by 3a and semantics of $\neg$)
            \end{enumerate}

            The second branch is:
            \begin{enumerate}[label=\arabic*b.]
                \item % 1b
                    $I \models \neg \top \land \neg \bot$ 
                \item % 2b
                    $I \models \neg \top$ (by 1b and semantics of $\land$)
                \item % 3b
                    $I \not\models \top$ (by 2b and semantics of $\neg$)
                \item % 4b
                    $I \models \top$ (Under any interpretation, $\top$ has value $\true$)
                \item % 5b
                    $I \models \bot$ (3b and 4b are contradictory)
            \end{enumerate}
            Thus both branches are closed, and thus $\top \leftrightarrow \neg \bot$ is valid.
        \item
            We will apply a strategy similar to that of Example 1.13.
            To prove $\bot \Leftrightarrow \neg \top$, we prove that $F: \bot \leftrightarrow \neg \top$ is valid.
            Suppose $F$ is not valid; there exists an interpretation $I$ such that $I \not\models F$.
            There are exactly two branches.

            The first branch is:
            \begin{enumerate}[label=\arabic*a.]
                \item % 1a
                    $I \models \bot \land \neg (\neg \top)$ (by semantics of $\leftrightarrow$)
                \item % 2a
                    $I \models \bot$ (by 1a and semantics of $\land$)
            \end{enumerate}

            The second branch is:
            \begin{enumerate}[label=\arabic*b.]
                \item % 1b
                    $I \models (\neg \bot) \land \neg (\neg \top)$ (by semantics of $\leftrightarrow$)
                \item % 2b
                    $I \models \neg\top$ (by 1b and semantics of $\land$)
                \item % 3b
                    $I \not\models \top$ (by 2b and semantics of $\neg$)
                \item % 4b
                    $I \models \top$ (by definition, P.7)
                \item % 5b
                    $I \models \bot$ (by 3b and 4b)
            \end{enumerate}
            Both of these two branches are closed; $F$ is valid.
            \addtocounter{enumi}{14} % Skip to (s)
        \item % (q)
        \item % (r)
        \item % (s)
            Let $F: (F_1 \rightarrow F_2) \leftrightarrow (\neg F_2 \rightarrow \neg F_1)$.
            We will use a truth table to show $F$ is valid.
            \begin{center}
                \begin{tabular}{ |c|c||c|c|c| }
                    \hline
                        $F_1$ & $F_2$ & $F_1 \rightarrow F_2$ & $\neg F_2 \rightarrow \neg F_1$ & $F$ \\ \hline
                        0 & 0 & 1 & 1 & 1 \\ \hline
                        0 & 1 & 1 & 1 & 1 \\ \hline
                        1 & 0 & 0 & 0 & 1 \\ \hline
                        1 & 1 & 1 & 1 & 1 \\ \hline
                \end{tabular}
            \end{center}
            \addtocounter{enumi}{4} % Skip to (x)
        \item % (x)
            Let $F: ((F_1 \rightarrow F_2) \land (F_1 \rightarrow F_3)) \leftrightarrow (F_1 \rightarrow F_2 \land F_3)$.
            We will use a truth table to show $F$ is valid.
            \begin{center}
                \begin{tabular}{ |c|c|c||c|c|c|c|c|c| }
                    \hline
                        $F_1$ & $F_2$ & $F_3$ & $F_1 \rightarrow F_2$ & $F_1 \rightarrow F_3$ & $(F_1 \rightarrow F_2) \land (F_1 \rightarrow F_3)$ &
                                $F_2 \land F_3$ & $F_1 \rightarrow F_2 \land F_3$ & $F$ \\
                    \hline
                        0 & 0 & 0 & 1 & 1 & 1 & 0 & 1 & 1  \\
                    \hline
                        0 & 0 & 1 & 1 & 1 & 1 & 0 & 1 & 1  \\
                    \hline
                        0 & 1 & 0 & 1 & 1 & 1 & 0 & 1 & 1  \\
                    \hline
                        0 & 1 & 1 & 1 & 1 & 1 & 1 & 1 & 1  \\
                    \hline
                        1 & 0 & 0 & 0 & 0 & 0 & 0 & 0 & 1  \\
                    \hline
                        1 & 0 & 1 & 0 & 1 & 0 & 0 & 0 & 1  \\
                    \hline
                        1 & 1 & 0 & 1 & 0 & 0 & 0 & 0 & 1  \\
                    \hline
                        1 & 1 & 1 & 1 & 1 & 1 & 1 & 1 & 1  \\
                    \hline
                \end{tabular}
            \end{center}
    \end{enumerate}
\end{exer}

\begin{exer}[1.3]
    $ $
    \begin{itemize}
        \item
            $\bot$ is equivalent to $\neg \top$.
            In other words, $\bot \Leftrightarrow \neg \top$ and that is proved in Exercise 1.2(b).
    \end{itemize}
\end{exer}

\begin{exer}[1.4]
    $ $
    \begin{itemize}
        \item
            We claim that $\neg F \Leftrightarrow F \nand F$.
            To prove that, we prove $\neg F \leftrightarrow F \nand F$ is valid.
            \begin{center}
                \begin{tabular}{ |c|c|c||c| }
                    \hline
                    $F$ & $\neg F$ & $F \nand F$ & $\neg F \leftrightarrow F \nand F$\\
                    \hline
                    0 & 1 & 1 & 1 \\
                    \hline
                    1 & 0 & 0 & 1 \\
                    \hline
                \end{tabular}
            \end{center}
        \item
            $F_1 \lor F_2 \Leftrightarrow (F_1 \nand F_1) \nand (F_2 \nand F_2)$.
            \begin{center}
                \begin{tabular}{ |c|c|c|c||c| }
                    \hline
                    $F_1$ & $F_2$ & $F_1 \nand F_1$ & $F_2 \nand F_2$ & $(F_1 \nand F_1) \nand (F_2 \nand F_2)$\\
                    \hline
                    0 & 0 & 1 & 1 & 0 \\
                    \hline
                    1 & 0 & 0 & 1 & 1 \\
                    \hline
                    0 & 1 & 1 & 0 & 1 \\
                    \hline
                    1 & 1 & 0 & 0 & 1 \\
                    \hline
                \end{tabular}
            \end{center}
    \end{itemize}
\end{exer}

\begin{exer}[1.5]
    $ $
    \begin{enumerate}[label=(\alph*)]
        \item
            By using the list of template equivalences on P.19, we can obtain the negation normal form of the original formula as following:
            \begin{itemize}
                \item
                    $F: \neg (P \rightarrow Q)$.
                \item
                    $F': \neg (\neg P \lor Q)$.
                \item
                    $F'': \neg (\neg P) \land \neg Q$.
                \item
                    $F''': P \land \neg Q$.
            \end{itemize}
            The only connectives in $F'''$ are $\neg, \land$, and $\lor$ and the negations appear only in literals.
            Thus $F'''$ is in NNF.
            Furthermore, $F'''$ is actually in CNF and DNF since it is the disjunction of one conjunction, and it is the conjunction of two clauses.
        \item
            $F': (\neg P \lor \neg Q) \land R$ is the NNF of $F$ that can be obtained using the same strategy as above.
            $F'$ is already in CNF.
            $F'': (\neg P \land R) \lor (\neg Q \land R)$ is an equivalent formula in DNF.
        \item
            $F:(Q \land R \rightarrow (P \lor \neg Q)) \land (P \lor R)$.
            \begin{itemize}
                \item
                    $F_1: (\neg (Q \land R) \lor (P \lor \neg Q)) \land (P \lor R)$.
                \item
                    $F_2: (\neg Q \lor \neg R \lor P \lor \neg Q) \land (P \lor R)$.
                \item
                    $F_3: (\neg Q \land (P \lor R)) \lor (\neg R \land (P \lor R)) \lor (P \land (P \lor R)) \lor (\neg Q \land (P \lor R))$.
                \item
                    $F_4: (\neg Q \land P) \lor (\neg Q \land R) \lor (\neg R \land P) \lor (\neg R \land R) \lor (P \land P) \lor (P \land R) \lor (\neg Q \land P) \lor (\neg Q \land R)$.
            \end{itemize}
            $F_2$ is in NNF and CNF, and $F_4$ is in DNF.
    \end{enumerate}
\end{exer}

\begin{exer}[1.6]
    $ $
    \begin{enumerate}[label=(\alph*)]
        \item
            $\bigwedge\limits_{v \in V} \bigvee\limits_{c \in C} P^c_v$.
        \item
            $\bigwedge\limits_{(v, w) \in E} \bigvee\limits_{c_1 \ne c_2 \in C} (\neg P_v^{c_1} \vee \neg P^{c_2}_w)$.
        \item
            $\bigwedge\limits_{(v, w) \in E} \bigvee\limits_{c \in C} (\neg P_v^c \vee \neg P^c_w)$.
        \item
            No clue.
            The problem statement seems too ambiguous.
        \item
            They are already in CNF.
            We have $N \cdot M$ variables and $N \cdot M + K \cdot M \cdot (M - 1) + K \cdot M$ clauses in the encoding above.
    \end{enumerate}
\end{exer}

\begin{exer}[1.7]
    $ $
    \begin{enumerate}[label=(\alph*)]
        \item
            $P_{(F)}$ contains 1 term.
            Each $\En(G)$ contains $1, 1, 2 \cdot 2 \cdot 3, 1, 1, 2 \cdot 2 \cdot 3, 3 \cdot 2 \cdot 2$ clauses.
            Thus the expansion would contain 1728 clauses, which is the product of the numbers of clauses.
        \item
            $ $
            \begin{enumerate}[label=(\roman*)]
                \item
                    $2^n$ clauses.
                \item
                    $\Rep(F_n) = P_{F_n}$, so it is just one clause.
                    We have the subformula set $S_{F_n} = \{ Q_1, \cdots, Q_n \} \cup \{ R_1, \cdots, R_n \} \cup \{ Q_i \land R_i \mid 1 \leq i \leq n \} \cup \{ (\bigvee_{i=1}^{k-1} Q_i \land R_i) \lor (Q_k \land R_k) \mid k = 2, \cdots, n \}$.
                    We will consider how many clauses $\En(G)$ has for each $G \in S_{F_n}$.
                    \begin{itemize}
                        \item
                            For each $Q_i$ and $R_i$, we have $\En(Q_i) = \En(R_i) = \top$.
                            This adds $2n$ clauses to $F'$.
                        \item
                            $\En(Q_i \land R_i)$ contains 3 clauses as defined on P.25.
                            This adds $3n$ clauses to $F'$.
                        \item
                            Each $\En((\bigvee_{i=1}^{k-1} Q_i \land R_i) \lor (Q_k \land R_k))$ contains 3 clauses as defined on P.25.
                            This adds $3(n - 1)$ clauses to $F'$.
                    \end{itemize}
                    Therefore, in total, $F'$ contains $1 + 2n + 3n + 3(n - 1) = 8n - 2$ clauses.
                    Note that $8n + 2 < 30n + 2$ where $30n + 2$ is the upper bound described on P.26.
                \item
                    When $n \leq 5$, $2^n$ is smaller than $8n - 2$.
                    For $n \geq 6$, $2^n$ is bigger than $8n - 2$.
            \end{enumerate}
    \end{enumerate}
\end{exer}

\begin{exer}[1.8]
    $ $
    \begin{enumerate}[label=(\alph*)]
        \item
            We will follow the format described in Example 1.30.
            Branching on $Q$ or $R$ will result in unit clauses; choose $Q$.
            Then $F\{ Q \mapsto \top \}: (P \lor \neg R) \land (R)$.
            $P$ appear only positively, so we consider $F\{ Q \mapsto \top, P \mapsto \top \}: R$.
            Then $R$ appears only positively, so the formula is satisfiable.
            In particular, $F$ is satisfied by interpretation
            \begin{align*}
                I: \{ P \mapsto \true, Q \mapsto \true, R \mapsto \true \}.
            \end{align*}
        \item
            Branching on $Q$ or $R$ will result in unit clauses.
            Choose $Q$.
            \begin{align*}
                F\{ Q \mapsto \top \}: (\neg P \lor \neg R) \land (R).
            \end{align*}
            $P$ appears only negatively.
            \begin{align*}
                F\{ P \mapsto \bot, Q \mapsto \top \}: R.
            \end{align*}
            $R$ appears only positively.
            Thus the interpretation $I: \{ P \mapsto \textsf{false}, Q \mapsto \textsf{true}, R \mapsto \textsf{true} \}$ satisfies $F$.
    \end{enumerate}
\end{exer}

